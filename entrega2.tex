\documentclass[12pt]{article}
\usepackage[spanish]{babel} % hace que, entre otras, el rompimiento de palabras se haga de acuerdo a las reglas del español
\selectlanguage{spanish}
\usepackage[utf8]{inputenc} % Incluye caracteres con tilde

\usepackage{pdfpages}
\usepackage{amsmath}
\usepackage{dsfont}

\usepackage{graphicx}   % permite insertar figuras
\usepackage{fancyhdr}   % para encabezado y pie de pagina
\usepackage{listings}   % para escribir código
%\usepackage{hyperref}   % crea links en el indice y notas al pie.
\usepackage{float}      % permite dejar las imagenes es una posición fija   

%\usepackage{anysize}   % Permite emplear cualquier medida de márgenes.
%\marginsize{3cm}{2cm}{3cm}{2cm} % Controla los márgenes {izquierda}{derecha}{arriba}{abajo}. 

\pagestyle{fancy}

\setlength{\headheight}{15.9pt} % configurar la altura del header

%\lhead{Introducción al trabajo de titulo}
\chead{CC6908}
\rhead{}%\includegraphics[width=30pt]{./images/LogoDCC.png}}

\usepackage{enumerate}
\usepackage{amsfonts} % para letras que representan conj numericos (R,Z,N,....)

    \usepackage{setspace} % para controlar el interlineado

    \newcommand{\anchoFirma}{1.5in}

    %\newcommand*{\firma}[6]{
        % 1° fila : líneas horizontales donde se firma
            %    \noindent\makebox[\anchoFirma]{\hrulefill}
        %        \hfill\makebox[\anchoFirma]{\hrulefill}
        %        \hfill\makebox[\anchoFirma]{\hrulefill}\newline
            % 2° fila: cargos
            %    \noindent\makebox[\anchoFirma][l]{Profesor Guía:}\hfill
            %    \noindent\makebox[\anchoFirma][l]{Profesor Co-Guía:}\hfill
            %    \noindent\makebox[\anchoFirma][l]{Profesor Integrante:}\hfill
            % 3° fila: nombres profes
            %    \noindent\makebox[\anchoFirma][l]{#1}\hfill
            %    \noindent\makebox[\anchoFirma][l]{#2}\hfill
            %    \noindent\makebox[\anchoFirma][l]{#3}\hfill
            %    \\[0.5in]
            % 4° fila: línea para firmar
            %    \noindent\makebox[\anchoFirma]{}
        %        \hfill\makebox[\anchoFirma]{\hrulefill}
        %        \hfill\makebox[\anchoFirma]{}\newline
            % 5° fila: cargo
            %    \noindent\makebox[\anchoFirma][l]{}\hfill
            %    \noindent\makebox[\anchoFirma][l]{Memorista:}\hfill
            %    \noindent\makebox[\anchoFirma][l]{}\hfill
            % 6° fila: nombre memorista
            %    \noindent\makebox[\anchoFirma][l]{}\hfill
            %    \noindent\makebox[\anchoFirma][l]{#4}\hfill
            %    \noindent\makebox[\anchoFirma][l]{}\hfill
            % 7° fila: mail
            %    \noindent\makebox[\anchoFirma][l]{}\hfill
            %    \noindent\makebox[\anchoFirma][l]{#5}\hfill
            %    \noindent\makebox[\anchoFirma][l]{}\hfill
            % 8° fila: celular
            %    \noindent\makebox[\anchoFirma][l]{}\hfill
            %    \noindent\makebox[\anchoFirma][l]{#6}\hfill
            %    \noindent\makebox[\anchoFirma][l]{}\hfil
            %}


            \newcommand*{\firma}[8]{
                \noindent\makebox[1.5in]{\hrulefill}
                \hfill\makebox[1.5in]{\hrulefill}
                \hfill\makebox[1.5in]{\hrulefill}
                \noindent\makebox[1.5in][l]{#1}\hfill
                    \noindent\makebox[1.5in][l]{#3}\hfill
                    \noindent\makebox[1.5in][l]{#5}\hfill
                    \noindent\makebox[1.5in][l]{#2}\hfill
                    \noindent\makebox[1.5in][l]{#4}\hfill
                    \noindent\makebox[1.5in][l]{#6}\hfill
                    \noindent\makebox[1.5in][l]{}\hfill
                    \noindent\makebox[1.5in][l]{}\hfill
                    \noindent\makebox[1.5in][l]{#7}\hfill
                    \noindent\makebox[1.5in][l]{}\hfill
                    \noindent\makebox[1.5in][l]{}\hfill
                    \noindent\makebox[1.5in][l]{#8}\hfill
            }



\begin{document}

\renewcommand{\tablename}{Tabla} % Cambia la palabra cuadro por Tabla cuando se introducen tablas
\renewcommand{\listtablename}{Índice de tablas} % Cambia titulo de indice de tabla
%================================================================================================================
\begin{titlepage}
\vspace{2cm}
\begin{center}
{\large
    UNIVERSIDAD DE CHILE
        \\[0.1cm]
        Facultad de Ciencias Físicas y Matemáticas
        \\[0.1cm]
        Departamento de ciencias de la computación
        \\[0.1cm]
        CC6908 Introducción al trabajo de titulo
        \\[0.1cm]
}
\end{center}
% Imagen de la portada
\begin{figure}[h] %[h] para here [b] para bottom [t] para top
\makebox[\textwidth]{\includegraphics[width=250pt]{./images/LogoFCFM.jpg}}
\end{figure}
\begin{center}
{\bf \huge \it Visualización de estructuras espaciales de desplazamiento a partir de datos de transporte público}
\\[1.0cm]
\today
\\[2.2cm]
\firma{Profesor Guía}{Marcela Munizaga}
{Profesor Co-guía}{Benjamín Bustos}
{Memorista}{Felipe A. Hernández G.} 
{fhernand@dcc.uchile.cl}{90977379}
\end{center}

\begin{center}
%Semestre: 02/2014\\
        %{\large \sc {\large Santiago de Chile}}
        \end{center}
        \end{titlepage}

        \newpage
        %================================================================================================================

        \section{Resumen ejecutivo}

        Aquí va el resumen ejecutivo

        \thispagestyle{empty} % para que no se numere esta pagina+

        \newpage
        %================================================================================================================
        \tableofcontents   % Indice

        %\listoftables % Indice de tablas

        %\listoffigures   % Indice de figuras

        \pagenumbering{roman} % las páginas correspondientes al índice serán enumeradas con números romanos
        \newpage

        \pagenumbering{arabic}


        \newpage
        %================================================================================================================
        \section{Introducción}

        %\linespread{1.7}\selectfont

        En los últimos años la utilización de tecnología en el transporte público ha ido en aumento debido a varios factores, mayor regulación, usuarios más exigentes, aumento en seguridad, etc. Lo anterior ha llevado al sistema público de transporte a implantar diversos dispositivos que permiten controlar los aspectos mas relevantes al momento de transportar una persona de un punto a otro.

        Dentro de las tecnologías más usadas podemos nombrar AVL (Automatic Vehicle Location), que permite conocer la posición geográfica de un vehículo en todo momento con un margen de error bajo y los sistemas AFC (Automated Fare Collection) que automatizan el proceso de pago, en particular, nos interesa el basado en tarjetas de pago, que albergan un chip que permite mantener un saldo para que sea utilizado al abordar un bus, esto implica la existencia (paralelamente) de dispositivos asociados a los buses o paraderos\footnote{Lugar físico donde un bus de transporte público se detiene para que personas ingresen y/o desciendan a el.} que permitan registrar el correcto descuento del valor asociado al pasaje, concepto que llamaremos validación.
% hablar de los APC (Automatic passenger counting)

    Transantiago\footnote{fue implementado a partir del año 2007.} es el sistema de transporte público de Santiago de Chile que implementa las tecnologías nombradas anteriormente, por lo que hoy en día se sabe que se realizan aproximadamente 6.000.000 de validaciones durante un día laboral\footnote{Lunes, martes, miércoles, jueves o viernes.}, lo que genera una cifra cercana a los 35.000.000 de transacciones a la semana (incluyendo sábado y domingo) con aproximadamente 3.000.000 de tarjetas de pago. Por otro lado, hay 80.000.000 de emisiones proveniente de la tecnología AVL del sistema. Al procesar estos datos en conjunto es posible identificar el paradero de origen, recorrido utilizado para desplazarse y el paradero de destino, este último requiere un procesamiento adicional basado en una metodología desarrollada por Munizaga y Palma \cite{Procesamiento_datos} que logra una identificación acertada en el 80\% de las validaciones.

    La estructura espacial moderna de las ciudades ha sido formada, en gran medida, por avances en transporte y comunicaciones \cite{Forma_ciudad_moderna}. La forma en la cual se mueven los habitantes de una ciudad ha ido modificando la estructura de esta, motivados por la transferencia de recursos como materiales, dinero, personas e información. Considerando una persona como un transportador de recursos de un área urbana a otra es que se identifican las siguientes estructuras espaciales urbanas \cite{Estructura_urbana}:

    \begin{itemize}
    \item Centros de flujo: Se refiere a las áreas que sirven para conectar otro par de áreas para transferencia de personas. Funcionan como puentes espaciales entre distintas áreas.
    \item Centros: Se refiere a áreas que concentran personas. Pueden diferir de los Centros de flujo, pero a menudo, son lo mismo.
    \item Bordes: Se refiere a límites socioeconómicos generados a partir de la agrupación de paraderos que divide la ciudad en pequeños barrios que llamamos comunidades.
    \end{itemize}

    %hablar a quien va dirigido la visualización. Personas de áreas que no se realacionan directamente con la computación. Gente de arquitectura y urbanismo.
    % mencionar que el objetivo de la visualización es comunicar información.

    Lo anterior se enmarca en la necesidad de comunicar esta información a personas sin una formación ingenieril dado que es un proyecto que abarcará muchos campos de investigación, como lo es la arquitectura o planificación urbana, por lo que es necesario transmitir datos de forma clara y concisa.

    De todo lo relatado podemos ver que hoy en día el sistema de transporte público de Santiago cuenta con una gran cantidad de datos pasivos por lo que existe una gran base de datos que mantiene un potencial de información que puede mejorar la planificación y operación del sistema, además de tener la potencialidad de detectar otras necesidades. Sin embargo, con las capacidades de procesamiento actuales no es posible obtener información que complemente los resultados de los indicadores usados actualmente.

    Según lo anterior, el problema que se busca resolver en esta memoria es interesante de abordar debido a que ayudará a entender la estructura espacial de los viajes realizados por la población y permitirá diseñar servicios pensando en las necesidades observadas de los usuarios. Esto ayudará a:

    \begin{itemize}
    \item Mejorar las posibilidades de realizar actividades en el entorno de la zona de residencia.
    \item Disminución de la demanda en los Centros de flujo.
    \item Disminución en el tiempo requerido para trasladarse hasta el punto de interés para una comunidad determinada.
    \item Mejorar las condiciones de viaje de grupos vulnerables.
    \end{itemize}

    % hablar de esto en la motivación
    %Para detectar estos elementos en la ciudad de Santiago haremos uso de los datos obtenidos de su  sistema de transporte público utilizando para ello el análisis de redes (grafos) y algoritmos de procesamiento de estructuras de grafos a gran escala como infomap, además de herramientas computacionales para diseñar una interfaz de visualización de los datos.

    %Hoy en día la capacidad de procesamiento de datos a nivel industrial está muy lejos de lograr los que tienen los centros más avanzados, y a su vez, estos no tienen todo el procesamiento que desean. Según lo anterior es que debieron formularse diversos métodos de procesamiento que permiten acotar el tiempo en el cuál se puede extraer información. 

    \newpage
    %================================================================================================================
    \section{Motivación}
    %confort (tiene que ver con que la gente se pueda mover dentro del bus o metro)
%confiabilidad (variabilidad de intervalos)


    Como hemos dado a conocer, existe una gran fuente de datos con mucha información pero que actualmente encuentra sus dificultades en el procesamiento y la forma en que puede ser comunicada. Por lo que una solución a este problema puede abrir las puertas a nuevas preguntas y según esto, nuevas investigaciones.

    También es interesante académicamente debido a la masividad de los datos, ya que se deberá implementar una estrategia de procesamiento que permita manejar millones de registros y además realizar análisis sobre estos que permitan comunicar información por medio de la visualización.

    %\newpage
    %================================================================================================================
    \section{Objetivos}

    \subsection{Objetivo General}

    ``Diseñar una herramienta que permita identificar y visualizar estructuras espaciales de movimiento en la ciudad de Santiago utilizando datos pasivos y masivos de transporte público.''
    \subsection{Objetivos específicos}

    \begin{enumerate}
    \item Construir modelo de red para la ciudad de Santiago. %basado en los paraderos del transporte público.
    \item Identificar patrones de viaje, centros y puntos de alto flujo de pasada.
    \item Desarrollar una herramienta que permita visualizar las estructuras espaciales.
    \end{enumerate}

    \newpage
    %================================================================================================================
    \section{Metodología}

    %===================================================================
    % Arreglar aquí
    %====================================================================

    Esta metodología está basada en una investigación publicada en la \textit{International Journal of Geographical Information Science} \cite{Estructura_urbana}, por lo que los procedimientos ya han sido probados en otro contexto, reduciendo de esta forma posibles inconvenientes que puedan ocurrir a lo largo del desarrollo de esta memoria. 

    Los datos a utilizar se han definido como los producidos en una semana de calendario (lunes a domingo). Estos ya se encuentran procesados según la metodología diseñada por Munizaga y Palma \cite{Procesamiento_datos}, por lo que la data corresponde a una tabla de una base de datos postgreSQL llamada \textit{tabla\_de\_etapas} donde cada fila representa una etapa de un viaje\footnote{un viaje puede tener una o más etapas.}. Según lo anterior la cantidad de datos a utilizar es de aproximadamente 35.000.000, que corresponde a la cantidad de etapas realizadas por el 80\% de las transacciones del sistema AFC\footnote{Automatic Fare Collection}.

    Dado lo anterior, el desarrollo de esta memoria considera la siguiente metodología de trabajo:

    \begin{enumerate}
    \item Investigación bibliográfica

    Se está realizando una recopilación y redacción de las ideas y estrategias más relevantes que aporten y justifiquen la base teórica de esta memoria.

    \item Estudio de la data.

    Se realizará un estudio de los datos existentes para comprender concretamente las bases de datos datos requeridos.  

    \item Definición de estrategia de pre-procesamiento de datos.

    Se investigará sobre las estrategias de pre-procesamiento y elegirá la que mejor se adapte en base al estudio realizado en el item anterior. Dentro de esta etapa se llevará a cabo la normalización y selección de la data para realizar los análisis.

    \item Construcción de la red de nodos.

    En esta etapa se realizará la construcción de un grafo dirigido con nodos a partir de la data estudiada.

    \item Análisis de la red
    \begin{enumerate}
    \item Definición de propiedades básicas.

    Aquí asociaremos un atributo de las estructuras urbanas a cada propiedad matemática de un grado a partir de las ideas obtenidas de la investigación bibliográfica.

    \item Definición de centralidades.
    \begin{enumerate}
    \item Centro de flujo.

    Se define el concepto de Centro de flujo en un grafo (\textit{betweenness centrality}) y se propone una fórmula para medirlo.

    \item Centro

    Aquí estudiaremos y definiremos la estrategia para detectar centros de la ciudad ocupando el algoritmo \textit{PageRank}.

    \end{enumerate}
    \item Estructura de comunidad.

    Para la detección de estructuras de comunidad se utilizará el software \textit{infomap}. 
    \end{enumerate}
    \item Análisis espacial
    \begin{enumerate}
    \item Interpolación espacial.

    Lo relevante de esta etapa es relacionar una zona geográfica a un paradero de bus de manera de poder particionar la ciudad.

    \item Cálculo estadístico.

    En esta etapa se realizará la asociación de las comunidades detectadas a las áreas geográficas establecidas en la interpolación espacial.

    \end{enumerate}
    \item Análisis de los resultados.

    Se estudiarán los resultados obtenidos.

    \item Definir visualizaciones y nivel de interactividad de cada una.

    A partir del punto anterior se definirán las visualizaciones a realizar y las posibles interacciones que puedan haber en cada una de ellas.

    \item Diseño de aplicación de visualización.

    Se desarrollará una aplicación que permita ver cada una de las implementaciones definidas en el punto anterior.

    \end{enumerate}

    Este trabajo será realizado a lo largo de 2 semestres (2014-2 y 2015-1) por lo que se dividirá de la siguiente forma:

    \begin{itemize}
    \item Semestre 2014-2
    \begin{enumerate}
    \item Investigación bibliográfica 
    \item Estudio de la data
    \item Definición de estrategia de pre-procesamiento de datos
    % mencionar que esta parte se intentará completar. El compromiso es medio.
    \end{enumerate}
    \item Semestre 2015-1
    \begin{enumerate}\setcounter{enumi}{3}
    \item Construcción de la red de nodos
    \item Análisis de la red
    \item Análisis espacial
    \item Análisis de los resultados
    \item Definir visualizaciones y nivel de interactividad de cada una
    \item Diseño de aplicación de visualización
    \end{enumerate}
    \end{itemize}

    Es importante decir que el punto 3 se espera abordarlo de manera parcial, realizando un acercamiento durante este período para luego finalizarlo previo inicio del segundo y así poder lograr el desarrollo del resto.


    \newpage
    %================================================================================================================

    \section{Revisión de antecendentes}
        \subsection{Análisis de bibliografía}
    
    El documento principal en el cuál está basado esta memoria es el artículo \textit{Detecting the dynamics of urban structure through spatial network analysis} \cite{Estructura_urbana} por lo que la metodología allí expuesta es utilizada en esta memoria para ser analizada con los datos locales de transporte público. El aporte de este documento al área radica en que provee un método cuantitativo para la detección de \textbf{Centro de flujo}, \textbf{Centros} y \textbf{Bordes} pudiendo detectar estructuras urbanas a partir de datos pasivos del transporte público, además de describir las técnicas que son aplicadas. Además establece una vinculación entre parámetros medibles con fenómenos urbanos reales, la cuál es posible aplicar a nuevas técnicas para detección de bordes basadas en otras metodologías que puedan aparecer en el futuro, permitiendo la comparación entre ellas.
    
	\subsubsection{Redes espaciales} 
	
	Una red espacial se entiende como un grafo cuyos vértices y arcos representan objetos geométricos del mundo real. Los nodos tienen una posición relativa a un sistema de referencia específico y los arcos expresan la forma física en que interactuan entre ellos, entendiendose esto último como la forma en que se puede llegar físicamente de uno a otro.
	Hace algunos años atrás los análisis espaciales urbanos se limitaban a utilizar el diseño de las calles en términos de su topología urbana, lo que tiene la limitante de no considerar la accesibilidad asociada a una calle como una característica dependiente de los movimientos humanos existentes. Además, este tipo de análisis tiende a ignorar los flujos urbanos y a justificar espacios y su forma en función de las propiedades de la red.
	En los últimos años, los estudios sobre estás redes comenzaron a incorporar medidas de peso que reflejan los datos de movimientos urbanos como flujos sobre la red pero concentrado en el sistema de tránsito, no sobre los espacios urbanos asociados a estos. 
	Además de los datos obtenidos a partir del transporte público han existido investigaciones basadas en otras fuentes de datos, como lo son los AVL basado en GPS (\textit{Global Positioning System}) o conjuntos de datos telefónicos.
	Este trabajo al igual que el artículo en el que está basado utiliza los datos de la tarjetas inteligentes del transporte público de santiago de Chile para llevar la creación de la red espacial con el objetivo de analizar estructuras de movimiento urbano.
	
    
    
    \subsubsection{Construcción y representación} 


	\subsubsection{Análisis complejo de redes}    
    \newpage
    
    
    La estructura espacial moderna de las ciudades ha sido formada, en gran medida, por avances en transporte y comunicaciones \cite{Forma_ciudad_moderna}. La forma en la cual se mueven los habitantes de una ciudad ha ido modificando la estructura de ésta, motivados por la transferencia de recursos como materiales, dinero, personas e información. Considerando una persona como un transportador de recursos de un área urbana a otra es que se identifican las siguientes estructuras espaciales urbanas \cite{Estructura_urbana}:

    \begin{itemize}
    \item Centros de flujo: Se refiere a las áreas que sirven para conectar otro par de áreas para transferencia de personas. Funcionan como puentes espaciales entre distintas áreas.
    \item Centros: Se refiere a áreas que concentran personas. Pueden diferir de los Centros de flujo, pero a menudo, son lo mismo.
    \item Bordes: Se refiere a límites socioeconómicos generados a partir de la agrupación de paraderos que divide la ciudad en pequeños barrios que llamamos comunidades.
    \end{itemize}
    
    
    
    Un sistema \textit{Automated Data Collection} (\textit{ADC}) se define como un conjunto tecnologías que permiten la captura de datos de un ecosistema. En el ámbito en el que se mueve esta memoria
    
    
	explicar que es ADC
	como está compuesto
	y como se ocupa en chile.    
    
    
    El sistema de transporte público de santiago de Chile utiliza actualmente un sistema \textit{Automated Data Collection} (\textit{ADC}) para capturar distintas variables del sistema como lo son, los cobros, posición de las micros, entre otros. Si bien existen 

    La captura masiva de datos pasivos a partir del transporte público es un hecho reciente y ha revolucionado la forma de analizar el transporte de una ciudad(y no tan solo eso). Anterior a este re
    La captura de datos hoy en día se hace a través de tecnología conocida como \textit{Automated Data Collection systems (ADC)}, dentro de ésta es posible agruparla en tres grupos

    \begin{itemize}
        \item 
    \end{itemize}

    Estos sistemas tienen grandes beneficios versus su 
    La documentación existente respecto a los sistemas de transporte público específican 3 áreas de inte AVL AFC


    Otro fenómeno interesante de evluar es la existencia de estructuras de small world
    
    
        \subsection{Análisis de datos}

	A continuación se detalla la forma en que funciona el sistema de transporte público que origina los datos, como se estima el proceso de bajada y por último los detalles de las tablas obtenidas posterior a la estimación.
	
			\subsubsection{Sistema tarifario}
En Santiago de Chile, el sistema AFC utilizado corresponde a las tarjetas de pago, donde en buses es el único método disponible y en metro es el más utilizado.

El sistema de pago en Transantiago es tal que cada pasajero paga una tarifa cuando accede al sistema, que permite a él o ella hacer tres transbordos dentro de las dos horas siguientes al pago. La estructura de pago es diferente entre el metro y buses. En buses, el único sistema de pago es mediante la tarjeta de pago (llamada comercialmente tarjeta bip!), mientras que en metro, es posible comprar un ticket o usar la tarjeta bip!, sin embargo el porcentaje de usuarios que compra el ticket es de aproximadamente 3\%.
%Por otro lado el pago en metro difiere, ligeramente más alto, para los horarios de alta demanda, entonces si un pasajero usa primero bus y luego transborda a metro, la diferencia entre ambas líneas de transporte es cobrada cuando la persona accede a éste último.

% actualizar datos
El sistema se caracteriza por tener cerca de 300 rutas de buses, 6000 buses disponibles agrupados en 6 operadores\footnote{Un operador es una empresa que se encarga de prestar servicio a una zona de santiago.}, aproximadamente 10.000 paraderos y una cifra que bordea los 150 kilómetros de rieles para el metro. 

Dada la alta demanda que ha experimentado, se crearon 150 zonas físicas llamadas ``zona paga" que están equipadas con sistema de pago (validadores) de vehículo donde el pasajero paga cuando entra a la estación, lo cual incrementa la eficiencia de las subidas a los buses pero genera una dificultad para determinar cual bus de todos los que allí se detienen tomó. Es importante decir que estas estaciones de buses operan durante los horarios de alta demanda en puntos de congestión identificados previamente.

Todas las transacciones bip! Son guardadas en una base de datos que contiene información sobre los operadores y el instante en que la transacción fue hecha. Lo anterior se lleva a cabo por cada pasajero acercando su tarjeta al validador, cuando ingresa al bus, zona paga o metro. Cada validador adjunta a cada transacción que realiza un id asociado a el y que está a su vez, asociado con un bus, estación de bus o metro. La información recolectada por cada transacción incluye: \textbf{id de la tarjeta}, \textbf{tipo}\footnote{puede ser comercial o estudiante.},  \textbf{código de bus o sitio donde se realizó la transacción}, \textbf{fecha y hora}, \textbf{monto de pago}. La posición espacial de la transacción puede ser conocida directamente para las zonas paga y las estaciones de metro dado que son conocidas con anterioridad, para las transacciones hechas en buses es posible pero no está disponible en la base de datos de transacciones.

%Todas las semanas se realizan alrededor de 35 millones de transacciones con cerca de 3 millones de tarjetas bip!. Por otro lado, 

Otra base de datos contiene información sobre la localización de todos los buses, como la \textbf{latitud}, \textbf{longitud}, \textbf{tiempo}, \textbf{fecha} y \textbf{velocidad instantánea}. Estos datos obtenidos a intervalos de 30 segundos y son asociados a cada bus a través de un número de placa y código de operador.

Cruzando la información de transacción y posición de las bases de datos por cada placa de bus o código de metro/estación de bus y tiempo, es posible identificar la localización espacial donde la transacción es realizada. Es así como en datos analizados del año 2009 y 2010 se logra una estimación exitosa en el 98,5\% y 99,9\% de los casos respectivamente.\cite{Procesamiento_datos}.

\subsubsection{Estimación de bajadas}

Como en el sistema de tarifa solo se validan las subidas, es necesario estimar los puntos de bajadas de las transacciones. Es aquí donde se utilizan una serie de supuestos para entender el comportamiento general de los usuarios dentro del sistema. Para lo anterior debemos definir lo que se entiende como un viaje, y \textit{Ortúzar and willumsen, 2011}\cite{Viaje} lo definen como:

\begin{center}
	\textit{``Un viaje se define como un movimiento desde un punto de origen a un punto de destino"}
\end{center} 

Esta definición origina la consideración de etapas dentro de un viaje, una etapa es la utilización de un servicio en particular (bus o metro). Es importante notar que no se consideran los cambios entre líneas de metro.

Básicamente la idea es seguir una cadena de viajes de una tarjeta e identificar la posición de bajada (de bus o metro) mirando la posición y el tiempo de la próxima subida de esta tarjeta. Esto es solamente posible cuando la actual y siguiente transacción tiene información de posición, la cual es tomada de la base de datos de localización automática de vehículos. En el caso de la última transacción del día, se asume que el destino es cercano al punto donde el primer viaje del día comienza, encontrando así un viaje cíclico diario para los usuarios particulares. Si hay solo un viaje por tarjeta, no es posible inferir con solo un día de información.

Los supuestos para llevar a cabo esta estimación son: 
\begin{itemize}
	\item Después de un viaje, el origen del siguiente determina el destino del primero. \cite{Supuesto_Barry}
	\item al final del día, los usuarios van a volver a la estación donde abordaron en el primer viaje del mismo día. \cite{Supuesto_Barry}
	\item Cada tarjeta corresponde a un usuario
\end{itemize}
, cada tarjeta corresponde a un usuario, 
, esto se hace considerando supuestos razonables basados en investigaciones previas o  

asumiendo que la próxima transacción bip! Ocurre después de la bajada.
Se asume que cada tarjeta corresponde a un usuario, entonces la tarjeta bip! y el usuario son usados indistintamente. Básicamente la idea es seguir una cadena de viajes de una tarjeta e identificar la posición de bajada (de bus o metro) mirando la posición y el tiempo de la próxima subida de esta tarjeta. Esto es solamente posible cuando la actual y siguiente transacción tiene información de posición, la cual es tomada de la base de datos de localización automática de vehículos. En caso de la última transacción del día, se asume que el destino es cercano al punto donde el primer viaje del día comienza, encontrando así un viaje cíclico diario para los usuarios particulares. Si hay solo un viaje por tarjeta, no es posible inferir con solo un día de información.

\subsubsection{Descripción de los datos}

A partir de la estimación de bajada se logra completar el proceso para obtener los datos de subida y bajada de cada etapa de un viaje. De lo anterior se obtiene una tabla con los siguientes atributos:

\begin{center}
  \begin{tabular}{| l | p{7cm} |}
    \hline
    Nombre campo & Descripción \\ \hline \hline
    tiempo\_subida & fecha y hora en que se realizó una validación en el sistema  \\ \hline
    id & identificador de la tarjeta que realizó la validación \\ \hline
    pago &  \\ \hline
    x\_subida &  \\ \hline
    y\_subida &  \\ \hline
    tipo\_transporte &  \\ \hline
    serviciosentidovariante &  \\ \hline
    tipo\_dia &  \\ \hline
    nviaje &  \\ \hline
    netapa &  \\ \hline
    x\_bajada &  \\ \hline
    y\_bajada &  \\ \hline
    tiempo\_bajada &  \\ \hline
    par\_subida &  \\ \hline
    par\_bajada &  \\ \hline
    comuna\_subida &  \\ \hline
    comuna\_bajada &  \\ \hline
    zona\_subida &  \\ \hline
    zona\_bajada &  \\ \hline
    distonroutesubidabajada &  \\ \hline
    disteuclidsubidabajada &  \\ \hline
    distonrouteparsubidaparbajada &  \\ \hline
    disteuclidparsubidaparbajada &  \\ \hline
    serv\_un\_zp2 &  \\ \hline
    sitio2 &  \\ \hline
    tiempo2 &  \\ \hline
    media\_hora &  \\ \hline
    tiempo\_trasbordo &  \\ \hline
    dist\_trasbordo &  \\ \hline
    tiempo\_caminata &  \\ \hline
    tiempo\_espera &  \\ \hline
    tiempo\_etapa &  \\ \hline
    tiempo\_espera\_estimado &  \\ \hline
    escolar &  \\ \hline
    factor\_exp\_etapa &  \\ \hline
    nbusesanteriores &  \\ \hline
    tiempo\_busanterior &  \\ \hline
    busanterior1er &  \\ \hline
    busanterior2do &  \\ \hline
    busanterior3er &  \\ \hline
    tipocorte &  \\ \hline
    proposito &  \\ \hline
  \end{tabular}
\end{center}

Es importante recalcar que esta tabla debe ser filtrada debido a que el proceso de estimación no logra detectar un destino para el 20\% de los casos por lo que se trabajará con el 80\% de la data de la tabla que corresponde a 


    %\newpage
    %================================================================================================================
    %\section{Conclusiones y trabajo futuro}
    
    
    \newpage
    %================================================================================================================
    \section{Bibliografía}

    \renewcommand{\section}[2]{} % Elimina el titulo "references"

    \bibliographystyle{plain}
    \bibliography{references}
    \nocite{libro_visualizacion} % agrega el libro en la bibliografia porque no está citado en ninguna parte.

    \end{document}


